%COMANDO PER AVERE IL CAPITOLO CON IL NOME CHE VOGLIAMO NOI

\titleformat{\chapter}
{\normalfont\bfseries\LARGE}{\thechapter}{18pt}{\LARGE}

%COMANDO PER LA SPAZIATURA DEI TITOLI DAL BORDO DEL FOGLIO

\titlespacing*{\chapter}{0cm}{0cm}{0.2cm}
\setlength{\headsep}{1.5cm} % spaziatura tra header e testo


%COMANDO PER LA SPAZIATURA DEL TESTO DAI BORDI LATERALI

\geometry{
	left=20mm,
	right=20mm,
}

%COMADNO PER AVERE L'INDICE DEL NOME CHE SI VUOLE

\renewcommand{\contentsname}{Indice}

%COMADNI PER OTTENERE SUBSUBSECTION NUMERATE E PRESENTI NELL'INDICE

\setcounter{tocdepth}{5}
\setcounter{secnumdepth}{5}

%COMANDI PER OTTENERE HEADER E FOOTER

\pagestyle{plain}

\fancypagestyle{plain}{
	\fancyhf{}
	\lhead{\includegraphics[width=2.5cm]{immagini/logo.png}}
	\chead{}
	\rhead{\fontsize{12}{10}Documento tecnico NTFLab}
	\lfoot{}
	\cfoot{\thepage\ di \pageref{LastPage}}
	\rfoot{}
}

\pagestyle{plain}

%COMANDI PER LINK

\hypersetup{
	colorlinks=true,
	linkcolor=black,
	filecolor=black,
	urlcolor=blue,
	citecolor=black,
}

\lstset{
	language=[LaTeX]Tex,%C++,
	keywordstyle=\color{RoyalBlue}, %\bfseries,
	basicstyle=\small\ttfamily,
	%identifierstyle=\color{NavyBlue},
	commentstyle=\color{Green}\ttfamily,
	stringstyle=\rmfamily,
	numbers=none, %left,%
	numberstyle=\scriptsize, %\tiny
	stepnumber=5,
	numbersep=8pt,
	showstringspaces=false,
	breaklines=true,
	frameround=ftff,
	frame=single
} 

\usepackage{listings}
\usepackage{color}
\definecolor{lightgray}{rgb}{.9,.9,.9}
\definecolor{darkgray}{rgb}{.4,.4,.4}
\definecolor{purple}{rgb}{0.65, 0.12, 0.82}

\lstdefinelanguage{JavaScript}{
	keywords={typeof, new, true, false, catch, function, return, null, catch, switch, var, if, in, while, do, else, case, break},
	keywordstyle=\color{blue}\bfseries,
	ndkeywords={class, export, boolean, throw, implements, import, this},
	ndkeywordstyle=\color{darkgray}\bfseries,
	identifierstyle=\color{black},
	sensitive=false,
	comment=[l]{//},
	morecomment=[s]{/*}{*/},
	commentstyle=\color{purple}\ttfamily,
	stringstyle=\color{red}\ttfamily,
	morestring=[b]',
	morestring=[b]"
}

\lstset{
	language=JavaScript,
	extendedchars=true,
	basicstyle=\footnotesize\ttfamily,
	showstringspaces=false,
	showspaces=false,
	numbers=left,
	numberstyle=\footnotesize,
	numbersep=9pt,
	tabsize=2,
	breaklines=true,
	showtabs=false,
	captionpos=b
}